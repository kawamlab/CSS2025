\documentclass{css2025}

\title{
	タイトル\textcolor{red}{(14pt,中央揃え)}
}

\subtitle{
    ―サブタイトル\textcolor{red}{(10.5pt,中央揃え)}―
}

\author[1]{〇新潟太郎}
\author[2]{長岡花子(発表者氏名,10.5pt,中央揃え,発表者には「○」を付ける)}
\affil[1)]{冬季大学\textcolor{red}{(発表者の所属)}}
\affil[2)]{スポーツ大学\textcolor{red}{(発表者の所属)}}


\begin{document}
\maketitle

\section{背景と目的}
\textcolor{red}{(本文はMS明朝,10.5pt,2段組)}冬季スポーツの代表的なスキーは,社会的特性,身体的・精神的効果の面から生涯スポーツとしての普及が試みられている.
そこで,本研究の目的は・・・を・・・することとした.

\section{方法}
調査方法は,北海道S市にあるTスキー場来場者500名を対象にアンケート調査を実施した.
調査内容は・・・.質問項目毎に得点化し,・・・・.統計処理では,・・・.

\begin{table}[h]
    \centering
    \caption{Sample Table}
    \begin{tabular}{|c|c|c|}
        \hline
        1 & 2 & 3 \\
        \hline
        4 & 5 & 6 \\
        \hline
        7 & 8 & 9 \\
        \hline
    \end{tabular}
    \label{tab:sample}
\end{table}

\section{結果}
・・・・・・・・・・・・・・・・・

・・・・・・・・・・・・・・

・・・・・・・・・・・・・・・・・・

% 図の挿入
\begin{figure}[htbp]
    \centering
    \fbox{\includegraphics[width=0.8\linewidth]{figures/dummy.pdf}}
    \caption{図の説明 \cite{cite:niigata2024} \cite{cite:青空文庫Aozo22:online}}
    \label{fig:fig1}
\end{figure}


\subfile{sections/references}

\end{document}